%Copyright  Jean-Philippe Eisenbarth
%This program is free software: you can 
%redistribute it and/or modify it under the terms of the GNU General Public 
%License as published by the Free Software Foundation, either version 3 of the 
%License, or (at your option) any later version.
%This program is distributed in the hope that it will be useful,but WITHOUT ANY 
%WARRANTY; without even the implied warranty of MERCHANTABILITY or FITNESS FOR A 
%PARTICULAR PURPOSE. See the GNU General Public License for more details.
%You should have received a copy of the GNU General Public License along with 
%this program.  If not, see <http://www.gnu.org/licenses/>.

%Based on the code of Yiannis Lazarides
%http://tex.stackexchange.com/questions/42602/software-requirements-specification-with-latex
%http://tex.stackexchange.com/users/963/yiannis-lazarides
%Also based on the template of Karl E. Wiegers
%http://www.se.rit.edu/~emad/teaching/slides/srs_template_sep14.pdf
%http://karlwiegers.com
\documentclass{scrreprt}
\usepackage{array}
\usepackage{listings}
\usepackage{underscore}
\usepackage[bookmarks=true]{hyperref}
\usepackage[utf8]{inputenc} 
\usepackage[english]{babel}
\usepackage{textcomp}
\usepackage{graphicx} %utilisation d'images
\usepackage{enumitem}
\hypersetup{
    bookmarks=false,    % show bookmarks bar?
    pdftitle={Software Requirement Specification},    % title
    pdfauthor={Jean-Philippe Eisenbarth},                     % author
    pdfsubject={TeX and LaTeX},                        % subject of the document
    pdfkeywords={TeX, LaTeX, graphics, images}, % list of keywords
    colorlinks=true,       % false: boxed links; true: colored links
    linkcolor=blue,       % color of internal links
    citecolor=black,       % color of links to bibliography
    filecolor=black,        % color of file links
    urlcolor=purple,        % color of external links
    linktoc=page            % only page is linked
}%
\def\myversion{1.0 }
\date{}
%\title
\usepackage{hyperref}
\begin{document}

\begin{flushright}
    \rule{15cm}{5pt}\vskip1cm
    \begin{bfseries}
        \Huge{SOFTWARE REQUIREMENTS\\ SPECIFICATION}\\
        \vspace{1.0cm}
        for\\
        \vspace{1.0cm}
        Gestionnaire de Stage Telecom Nancy\\
        \vspace{1.0cm}
        \LARGE{version \myversion approved}\\
        
        \vspace{1.0cm}
        Prepared by \\ 
        \vspace{1.0cm}
        \begin{flushleft}
        Alexandre Chichmanian \\ Gauthier Zambaux \\ Mouchid Waidi \\ Guillaume Garcia\\
        \end{flushleft}
        \vspace{1.0cm}
        Los Pollos\\
        \vspace{1.0cm}
        \today\\
    \end{bfseries}
\end{flushright}

\renewcommand{\contentsname}{Contenu}
\tableofcontents


\chapter*{Revision History}

\begin{center}
    \begin{tabular}{|c|c|c|c|}
        \hline
	    Name & Date & Reason For Changes & Version\\
        \hline
	    21 & 22 & 23 & 24\\
        \hline
	    31 & 32 & 33 & 34\\
        \hline
    \end{tabular}
\end{center}

\chapter{INTRODUCTION}

\section{A propos de ce document}
\hspace{1cm}Ce document a pour objectif de présenter l'application de gestion des stages de Télécom Nancy, d'en donner les acteurs et les fonctionnalités ainsi que les contraintes qui y sont assiciées. Il présente par ailleurs aussi l'interface graphique de l'application.

\section{Portée du document}
\hspace{1cm}L'application présentée ici a pour objectif de facilité la gestion des stages des étudiants de Télécom Nancy.\\

\hspace{0.7cm}Du point de vue administratif, elle facilite la gestion des stages en permettant la création de bases de données évitant la recopie manuelle d'informations. Les conventions de stage sont notamment générée automatiquement.\\

\hspace{0.7cm}Par le biais de cette application, les étudiants ont la possibilités de saisir directement les informations relatives à leur stage, d'obtenir les informations dont ils peuvent avoir besoin (par exemple avec le livret de l'élève) et de rendre leur rapport de stage en ligne.\\

\hspace{0.7cm}Finalement, l'application permet aux professeurs référents de communiquer avec les étudiants, de récupérer les rapports à corriger et de valider ou non les stages de élèves.

\section{Public concerné et vue d'ensemble du document }

\hspace{1cm}Ce cahier des charges visent avant tout les instigateurs et responsables du projet ainsi que les développeurs. Par extension, il peut aussi être utile pour les futurs utilisateurs de l'application, c'est-à-dire les étudiants, les professeurs et l'administration de Télécom Nancy.\\

\hspace{0.7cm}La première partie de ce document introduit les objectifs principaux de l'applications ainsi que les personnes visées. Elle se charge aussi de définir les conventions de rédaction du cahier de charge.\\

\hspace{0.7cm}La deuxième partie se concentre sur une decription plus détaillée des fonctionnalités attendues et de l'environnement d'évolution de l'application, notamment son adaptabilité aux utilisateurs.\\

\hspace{0.7cm}La troisième partie présente l'interface utilisateur ainsi que les différentes contraintes auquelles le logiciel doit se plier.\\

\hspace{0.7cm}La quatrième partie présente les besoins non fonctionnels non présentés précédemment.\\

\hspace{0.7cm}Finalement, les annexes présentent le dictionnaire des données et le diagramme entités-associations.

\section{Définitions, acronymes et abréviations}
\hspace{1cm}Voici une liste des différentes abréviations utlisées dans ce document :
\begin{description}
\item[CAS :] Central Authentification Service service d'authentification de l'UL
\item[UL :] Université de Lorraine
\end{description}

\section{Conventions de rédaction du document}
$<$List any other documents or Web addresses to which this SRS refers. These may 
include user interface style guides, contracts, standards, system requirements 
specifications, use case documents, or a vision and scope document. Provide 
enough information so that the reader could access a copy of each reference, 
including title, author, version number, date, and source or location.$>$

\section{Références et remerciements}
$<$List any other documents or Web addresses to which this SRS refers. These may 
include user interface style guides, contracts, standards, system requirements 
specifications, use case documents, or a vision and scope document. Provide 
enough information so that the reader could access a copy of each reference, 
including title, author, version number, date, and source or location.$>$


\chapter{DESCRIPTION GLOBALE }

\section{Perspective du produit}
$<$Describe the context and origin of the product being specified in this SRS.  
For example, state whether this product is a follow-on member of a product 
family, a replacement for certain existing systems, or a new, self-contained 
product. If the SRS defines a component of a larger system, relate the 
requirements of the larger system to the functionality of this software and 
identify interfaces between the two. A simple diagram that shows the major 
components of the overall system, subsystem interconnections, and external 
interfaces can be helpful.$>$

\section{Fonctionnalités du produit}
$<$Summarize the major functions the product must perform or must let the user 
perform. Details will be provided in Section 3, so only a high level summary 
(such as a bullet list) is needed here. Organize the functions to make them 
understandable to any reader of the SRS. A picture of the major groups of 
related requirements and how they relate, such as a top level data flow diagram 
or object class diagram, is often effective.$>$

\section{Utilisateurs}
$<$Identify the various user classes that you anticipate will use this product.  
User classes may be differentiated based on frequency of use, subset of product 
functions used, technical expertise, security or privilege levels, educational 
level, or experience. Describe the pertinent characteristics of each user class.  
Certain requirements may pertain only to certain user classes. Distinguish the 
most important user classes for this product from those who are less important 
to satisfy.$>$

\section{Environnement d'exécution}
$<$Describe the environment in which the software will operate, including the 
hardware platform, operating system and versions, and any other software 
components or applications with which it must peacefully coexist.$>$

\section{Contraintes de conception et d'implémentation}
$<$Describe any items or issues that will limit the options available to the 
developers. These might include: corporate or regulatory policies; hardware 
limitations (timing requirements, memory requirements); interfaces to other 
applications; specific technologies, tools, and databases to be used; parallel 
operations; language requirements; communications protocols; security 
considerations; design conventions or programming standards (for example, if the 
customerés organization will be responsible for maintaining the delivered 
software).$>$

\section{Manuel utilisateur}
$<$List the user documentation components (such as user manuals, on-line help, 
and tutorials) that will be delivered along with the software. Identify any 
known user documentation delivery formats or standards.$>$


\section{Hypothèses et dépendances}
$<$List any assumed factors (as opposed to known facts) that could affect the 
requirements stated in the SRS. These could include third-party or commercial 
components that you plan to use, issues around the development or operating 
environment, or constraints. The project could be affected if these assumptions 
are incorrect, are not shared, or change. Also identify any dependencies the 
project has on external factors, such as software components that you intend to 
reuse from another project, unless they are already documented elsewhere (for 
example, in the vision and scope document or the project plan).$>$


\chapter{BESOINS SPÉCIFIQUES}

\section{Besoins externes}
\subsection{Interfaces utilisateur}
Que ce soit les étudiants, les professeurs, le directeur, la secrétaire des stages, il faut que l'utilisateur puisse se connecter de façon sécurisé. Elle se fait donc par le CAS de l'Université de Lorraine.
\vspace{0.5cm}
\begin{center}
\includegraphics[scale=0.4]{image/CAS.png}
\end{center}
\vspace{3cm}
\begin{flushleft}
Une fois la connexion établie, l'étudiant peut télécharger les documents dont il a besoin pour son stage avec en vert, les éléments qu'il peut télécharger, et en rouge, ceux qu'il ne peut pas.
\end{flushleft}
\begin{center}
\includegraphics[scale=0.65]{image/DownloadEleve.png}
\end{center}
\vspace{8cm}
\begin{flushleft}
Il peut aussi trouver les fichiers ou données qu'il doit envoyer sur la plateforme. Avec un vert les fichiers qu'il à déjà envoyés et en rouge, les fichiers non envoyés.
\end{flushleft}
\begin{center}
\includegraphics[scale=0.65]{image/uploadeleve.png}
\end{center}
\begin{flushleft}
\vspace{8cm}
Pour ce qui concerne l'envoi des informations relatives au stage, l'élève doit envoyer un certain nombre d'informations nécessaires à la génération automatique de la fiche.
\end{flushleft}
\begin{center}
\includegraphics[scale=0.65]{image/inforelativesstages.png}
\end{center}

\section{Besoins fonctionnels}
$<$Describe the logical and physical characteristics of each interface between 
the software product and the hardware components of the system. This may include 
the supported device types, the nature of the data and control interactions 
between the software and the hardware, and communication protocols to be 
used.$>$

\section{Besoins comportementaux}
$<$Describe the connections between this product and other specific software 
components (name and version), including databases, operating systems, tools, 
libraries, and integrated commercial components. Identify the data items or 
messages coming into the system and going out and describe the purpose of each.  
Describe the services needed and the nature of communications. Refer to 
documents that describe detailed application programming interface protocols.  
Identify data that will be shared across software components. If the data 
sharing mechanism must be implemented in a specific way (for example, use of a 
global data area in a multitasking operating system), specify this as an 
implementation constraint.$>$
\subsection{Diagrammes de séquence des cas d'utilisation}
%AUTENTIFICATION :
\begin{center}
	\includegraphics[scale=0.55]{image/authentification.png}
\end{center}
%\begin{center}
%	\textbf{Figure 3.3.1.} S'autentifier
%\end{center}
\hspace{1cm}Les utilisateurs de l'application doivent s'y authentifier à son ouverture. Ils utilisent pour cela leurs identifiants de l'Université de Lorraine.

%ACCES AU LIVRET DE L'ELEVE :
\begin{center}
	\includegraphics[scale=0.55]{image/accesLivretEleve.png}
\end{center}
\hspace{1cm}Les étudiants doivent pouvoir télécharger le livret de l'élève depuis l'application ; pour ce faire, ils doivent d'abord se connecter au service.

%SAISIE DES INFORMATIONS RELATIVES AU STAGE :
\begin{center}
	\includegraphics[scale=0.4]{image/saisieInfosStage.png}
\end{center}
\hspace{1cm}Plutôt que de remplir la fiche d'informations relatives au stage et la convention qui doivent ensuite être recopiées manuellement par le secrétarie des stages, l'étudiant a la possibilité d'entrer directement ces informations dans l'application qui émet ensuite automatiquement la convention à signer.

%DEPOSE DU RAPPORT :
\begin{center}
	\includegraphics[scale=0.55]{image/deposeRapport.png}
\end{center}
\hspace{1cm}L'une des fonctionnalités principales de l'application est de permettre à l'étudiant de déposer son rapport de stage pour que celui-ci soit corrigé. L'application lui indique si la réception du document a eu lieu ou pas.

%OBTENTION DES DATES DE SOUTENANCE :
\begin{center}
	\includegraphics[scale=0.55]{image/obtentionDatesSoutenance.png}
\end{center}
\hspace{1cm}Les étudiants peuvent obtenir les dates de soutenance qui leur sont affectées par l'intermédiaire de l'application après authentification.

%TELECHARGEMENT DE LA FICHE D'EVALUATION :
\begin{center}
	\includegraphics[scale=0.55]{image/telechargementFicheEvaluation.png}
	\includegraphics[scale=0.55]{image/envoiFicheEvaluation.png}
\end{center}
\hspace{1cm}L'application permet aux étudiant de télécharger la fiche d'évaluation par le maître de stage pour la lui faire remplir. Après cette opération, ils ont la possibilité de déposer la fiche remplie sur la plateforme.

%MISE EN LIGNE DE DOCUMENTS PAR LE SECRETARIAT DES STAGES :
\begin{center}
	\includegraphics[scale=0.55]{image/miseEnLigneDocumentsParSecretaire.png}
\end{center}
\hspace{1cm}Pour que les étudiants puissent télécharger les documents dont ils ont besoin pour le bon déroulement de leur stage, le secrétariat des stages doit pouvoir les mettre en ligne, accessibles par le biais de l'application développée ici.

%ACCES AUX DONNEES MISES EN LIGNE PAR LES ETUDIANTS PAR LE SECRETARIAT DES STAGES :
\begin{center}
	\includegraphics[scale=0.55]{image/accesDonneesParSecretaire.png}
\end{center}
\hspace{1cm}Après que les étudiants ont entré les informations relatives au stage dans l'application, le secrétariat des stages y a accès ainsi qu'a la convention générée automatiquement.

%MODIFICATION DE L'ANNEE D'APPARTENANCE D'UN ETUDIANT PAR LE SECRETARIAT DES STAGES :
\begin{center}
	\includegraphics[scale=0.55]{image/modifAnneeParSecretaire.png}
\end{center}
\hspace{1cm}Le secrétariat des stages a la charge de modifier l'année d'appartenance d'un étudiant lorsque son stage est validé par le professeur référent.

%ACCES AUX DONNEES MISES EN LIGNE PAR LES ETUDIANTS PAR LE PROFESSEUR RESPONSABLE :
\begin{center}
	\includegraphics[scale=0.55]{image/accesDonneesParProfesseur.png}
\end{center}
\hspace{1cm}Comme pour le secrétariat des stages, après que les étudiants ont entré les informations relatives au stage dans l'application, leur professeur répérent y a accès.

%VALIDATION OU NON DU STAGE PAR LE PROFESSEUR :
\begin{center}
	\includegraphics[scale=0.55]{image/validationStage.png}
\end{center}
\hspace{1cm}Après qu'il a corrigé le rapport de stage de l'étudiant et qu'il a consulté la fiche d'évaluation du stage par l'encadrant, le professeur référent a la possibilité de valider ou non le stage de l'élève.

\subsection{Diagrammes de cas d'utilisation}
%DIAGRAMME DE CAS D'UTILISATION DE L'ETUDIANT :
\subsubsection{Diagrammes de cas d'utilisation de l'étudiant}
\begin{center}
	\includegraphics[scale=0.55]{image/casutilisationetudiant.png}
\end{center}
%DESCRIPTION DE CAS D'UTILISATION ANNEE EN COURS :
\subsubsection{Description de cas d'utilisation de l'année en cours}
\begin{center}
	\includegraphics[scale=0.55]{image/descriptiondecasanneencours.png}
\end{center}
%DIAGRAMME DE CAS D'UTILISATION DE L'ENCADREUR :
\subsubsection{Diagrammes de cas d'utilisation de l'encadreur}
\begin{center}
	\includegraphics[scale=0.55]{image/casutilisationencadreur.png}
\end{center}
%DIAGRAMME DE CAS D'UTILISATION DU DIRECTEUR :
\subsubsection{Diagrammes de cas d'utilisation du directeur}
\begin{center}
	\includegraphics[scale=0.55]{image/casutilisationdudirecteur.png}
\end{center}
%DIAGRAMME DE CAS D'UTILISATION DE L'ADMINISTRATEUR :
\subsubsection{Diagrammes de cas d'utilisation de l'administrateur}
\begin{center}
	\includegraphics[scale=0.55]{image/casutilisationdeadministrateur.png}
\end{center}


\chapter{AUTRES BESOINS NON FONCTIONNELS}
$<$This template illustrates organizing the functional requirements for the 
product by system features, the major services provided by the product. You may 
prefer to organize this section by use case, mode of operation, user class, 
object class, functional hierarchy, or combinations of these, whatever makes the 
most logical sense for your product.$>$

\section{Besoins liés à la performance}
$<$Dont really say System Feature 1 State the feature name in just a few 
words.$>$

\subsection{Besoins liés à la sécurité}
$<$Provide a short description of the feature and indicate whether it is of 
High, Medium, or Low priority. You could also include specific priority 
component ratings, such as benefit, penalty, cost, and risk (each rated on a 
relative scale from a low of 1 to a high of 9).$>$

\subsection{Besoins liés à la qualité du produit }
$<$List the sequences of user actions and system responses that stimulate the 
behavior defined for this feature. These will correspond to the dialog elements 
associated with use cases.$>$





\chapter{AUTRES BESOINS}



\
\section{ANNEXE A: Dictionnaire de données}

\begin{center}
	\includegraphics[scale=0.45]{image/diagentiassoc.png}
	\textbf{Figure 5.1.1.} Diagramme entités-associations
\end{center}

\vspace {4cm}
\begin{center}
\textbf 
{Étudiants}
\vspace {0,5cm}

\begin{tabular}{|c|c|c|c|}
  \hline
  \textbf {Champ} & \textbf {Type} & \textbf {Description} & \textbf {Commentaires} \\
  \hline
  id_etudiant & varchar2 & identifiant de l'étudiant UL & clé primaire\\
  \hline
  nom & varchar2 & nom de l'élève &  \\
  \hline
  prénom & varchar2 & prénom(s) de l'étudiant &  \\
  \hline
  sexe & varchar2 & sexe de l'étudiant &  \\
  \hline
  date_naissance & date & date de naissance de l'étudiant &  \\
  \hline
  ine & int & numéro INE de l'étudiant & clé primaire \\
  \hline
  annee_etude & varchar2 & l'année en cours de l'élève (1A, 2A, 3A) &  \\
  \hline
  adresse_mail & varchar2 & adresse mail de l'étudiant & clé primaire\\
  \hline
  telephone & varchar2 & numéro de téléphone de l'étudiant &  \\
  \hline
  groupe & int & groupe de l'étudiant &  \\
  \hline
  mot_de_passe & varchar2 & mot de passe de l'étudiant &  \\
  \hline
\end{tabular}

\vspace {1cm}
\textbf 
{Soutenances}
\vspace {0,5cm}

\begin{tabular}{|c|c|c|c|}
  \hline
  \textbf {Champ} & \textbf {Type} & \textbf {Description} & \textbf {Commentaires} \\
  \hline
  id_soutenance & int & id de la soutenance & clé primaire\\
  \hline
  id_etudiant & varchar2 & identifiant de l'étudiant UL & clé primaire\\
  \hline
  id_professeur & varchar2 & id du professeur assistant à la soutenance &  \\
  \hline
  id_professeur_referent & varchar2 & identifiant du professeur référent &  \\
  \hline
  date & date & date de soutenance &  \\
  \hline
  heure_debut & time & heure de début de soutenance &  \\
  \hline
  heure_fin & time & heure de fin de soutenance &  \\
  \hline
  salle_soutenance & varchar2 & salle accueillant la soutenance &  \\
  \hline
\end{tabular}

\vspace {1cm}
\textbf 
{Service Informatique}
\vspace {0,5cm}

\begin{tabular}{|c|c|c|c|}
  \hline
  \textbf {Champ} & \textbf {Type} & \textbf {Description} & \textbf {Commentaires} \\
  \hline
  id_informaticien & varchar2 & identifiant de l'informaticien UL & clé primaire\\
  \hline
  nom & varchar2 & nom de l'informaticien & \\
  \hline
  prenom & varchar2 & prénom(s) de l'informaticien &  \\
  \hline
  sexe & varchar2 & sexe de l'informaticien &  \\
  \hline
  adresse_mail & varchar2 & adresse mail de l'informaticien & clé primaire  \\
  \hline
  mot_de_passe & varchar2 & mot de passe de l'informaticien &  \\
  \hline
  telephone & varchar2 & numéro de téléphone du service informatique &  \\
  \hline
\end{tabular}

\vspace {2cm}
\textbf 
{Directeur}
\vspace {0,5cm}

\begin{tabular}{|c|c|c|c|}
  \hline
  \textbf {Champ} & \textbf {Type} & \textbf {Description} & \textbf {Commentaires} \\
  \hline
  id_directeur & varchar2 & identifiant du directeur UL & clé primaire\\
  \hline
  nom & varchar2 & nom du directeur & \\
  \hline
  prenom & varchar2 & prénom(s) du directeur &  \\
  \hline
  sexe & varchar2 & sexe du directeur &  \\
  \hline
  adresse_mail & varchar2 & adresse mail du directeur & clé primaire  \\
  \hline
  mot_de_passe & varchar2 & mot de passe du directeur &  \\
  \hline
  telephone & varchar2 & numéro de téléphone du directeur &  \\
  \hline
\end{tabular}

\end{center}

\end{document}
